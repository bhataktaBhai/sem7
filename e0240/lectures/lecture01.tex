\chapter*{The course}
\lecture{2025-08-06}{}

\textbf{Instructor:} Shubhada Agrawal
\\[1em]
\textbf{Resources:}
\begin{enumerate}
    \item Tor Lattimore and Csaba Szepesvári. Bandit Algorithms
    \item Aleksandrs~Slivkins. Introduction to Multi-Armed Bandits
\end{enumerate}
\vspace{1em}
\textbf{Evaluation:}
\begin{itemize}
    \item[(50\%)] Homework
    \item[(30\%)] Final exam
    \item[(20\%)] Literature review + presentations
\end{itemize}

Multi-armed bandits is a special case of reinforcement learning.
The setting is the following.
We are the learner.
There are $k$ arms, labelled $1, 2, \dots, k$.
These have associated unknown probability distributions
$P_1, P_2, \dots, P_k$.
At each time step $t$, we choose one of the arms to pull, and receive a
\emph{reward} $X_t \sim P_{\text{chosen arm}}$.
The choice may depend on the previous rewards.

This problem was first studied in the 1930s by Thompson in an attempt to
dynamize clinical trials---if one drug is shown to be adverse quickly, why
put more patients through suffering?
\begin{examples}
    \item \textbf{News websites}.
        Whenever a user visits a website, the action is choosing a header
        to display.
        The reward is $1$ if the user clicks and $0$ otherwise.
    \item \textbf{Dynamic pricing.}
        The action is to decide a price $p$ to place.
        The reward is $p$ if a purchase is made and $0$ otherwise.
    \item \textbf{Investments.}
        Choose a stock to invest in.
        The reeward is the change in stock price.
\end{examples}
In general, there is a trade-off between exploration and exploitation.
At each time step, we could repeatedly turn the arm that seems best till
now, but we must try different options many times to get to know the best
arms.

The problem is theoretically rich, with connections to probability theory,
concentration of measure, complexity theory, algorithm design, everything.

\section*{Relaxations}

\subsection*{Feedback}
The above described model may not model the feedback in all scenarios.
That is the \emph{bandit feedback} setting.
But in the dynamic pricing example, we learn about prices other that the
price $p$ chosen.
\begin{itemize}
    \item If the customer does not make a purchase, we learn that they
        wouldn't have at all higher prices.
    \item If the customer does make a purchase, we learn that they would
        have made a purchase at all lower prices.
\end{itemize}
This is a \emph{partial feedback} setting.

In the investments example, we get feedback for all stocks.
This is a \emph{full feedback} setting, but is largely out of scope for
this course and is more closely related to online learning.

\subsection*{Contextual bandits}
The algorithm could be receiving some additional information at each time
step.
In the example of news websites, each user comes with a demographic profile.

\subsection*{Time homogeneity}
In the above model, the probability distributions $P_1, P_2, \dots, P_k$ are
independent of time.
Choosing from the same arm at different time steps gives iid rewards.

We could instead assume that the distributions change over time in a
Markovian manner, but independent of each other.

We might have some structural assumptions on the distributions, such as in
the dynamic pricing example.

\subsection*{Constraints}
We may be attempting to maximize positive medical results subject to the
constraint that nobody fucking dies.

We'll be spending roughly half the course each on the regret minimization
probelm and the best arm identification problem, respectively.
