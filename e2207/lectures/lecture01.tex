\chapter*{The course}
\lecture{2025-08-05}{}

\textbf{Resources:}
\begin{enumerate}
    \item
\end{enumerate}
\vspace{1em}
\textbf{Evaluation:}
\begin{itemize}
    \item
\end{itemize}

\begin{question}
    Toss a fair coin $10000$ times.
    How many heads do you expect to see?
\end{question}
Our expectation is $5000$.

\begin{theorem*}[Weak law of large numbers] \label{thm:wlln}
    If $X_1, X_2, \dots$ are iid random variables with finite mean and
    variance, then $\frac1n \sum_{i=1}^n X_n \pto \E[X_1]$.
    That is, for any $\epsilon > 0$, \[
        \lim_{n \to \infty}
            \P\set*{\abs*{\frac1n \sum_{i=1}^n X_i - \E[X_1]} > \epsilon} = 0.
    \]
\end{theorem*}
The finite variance assumption is not necessary.

For a fixed $\eps$, the error bound on $S_n$ is $O(n)$.

\begin{theorem}[Central limit theorem] \label{thm:clt}
    If $X_1, X_2, \dots$ are iid random variables with finite mean $\mu$
    and variance $\sigma$, then \[
        \frac1{\sqrt n} (\sum_{i=1}^n (X_i - \mu)) \dto N(0, \sigma^2).
    \]
\end{theorem}
Using this, once more, we have that the number of heads is
$5000 \pm \sqrt{10000 (\frac14)} Q^{-1}(0.005)$ with probability
at least $0.99$, if we assume that $10000 = \infty$,
where $Q(x) = \P(Z \ge x)$ for $Z \sim N(0, 1)$.
$Q^{-1}(0.005) \approx \sqrt(-\log 0.005)$.
The uncertainty here happens to be around $115$ tosses.

We will eventually see the Berry-Esseen theorem.
\begin{theorem}[Berry-Esseen inequality] \label{thm:berry}
    If $X_1, X_2, \dots$ are iid with zero mean and finite \textbf{third}
    moment, then there exists a constant $C$ such that for each
    $n$ and $\eps$, \[
        \abs*{\P \set*{\frac1{\sigma \sqrt n} \sum_{i=1}^n X_i > \eps}
            - Q(\eps)} \le \frac{C}{\sqrt n},
    \]
\end{theorem}
The constant here is known to be between $0.4$ and $0.5$.

\begin{theorem}[Large deviations] \label{thm:ldev}
    For every interval $A \subseteq \R$, \[
        \lim_{n \to \infty} -\frac1n \log \P\set*{\frac1n \sum_{i=1}^n X_i \in A}
            = \inf_{x \in A} I(x),
    \] where $I(\cdot)$ is the \emph{rate function}.
\end{theorem}
Thus assuming $10000 = \infty$, we can say that for $x > \frac12$, \[
    \P(S_n > n x) = \exp(-n I(x)).
\] We have written $I(x)$ in place of $\inf_{y > x} I(y)$ because $I$
happens to be decreasing in this case.

These were largely asymptoptic results.
The two simplest concentration inequalities are non-asymptotic.
\begin{theorem}[Markov's inequality] \label{thm:markov}
    If $X$ is a non-negative random variable, then for any $t > 0$, \[
        \P(X \ge t) \le \frac{\E[X]}{t}.
    \]
\end{theorem}
\begin{theorem}[Chebyshev's inequality] \label{thm:cheby}
    If $X$ is a random variable with finite variance,
    then for any $\delta > 0$, \[
        \P(\abs{X - \mu} \ge \delta) \le \frac{\sigma^2}{\delta^2}.
    \]
\end{theorem}
Chebyshev's inequaity gives that the number of heads is
$5000 \pm \sqrt{\frac{2500}{0.01}}$ with probability at least $0.99$.
(The uncertainty is $500$ tosses.)

In general, CLT and Chebyshev's inequality give results \begin{align*}
    &n \E X_1 \pm \sqrt{n \Var X_1} \sqrt{\log \frac2\delta}, \\
    \text{and } &n \E X_1 \pm \sqrt{n \Var X_1} \sqrt{\frac1\delta}
\end{align*}
with probability at least $1 - \delta$, repectively.

The Chernoff bound, for iid Bernoulli random variables, gives \[
    n \E X_1 \pm C \sqrt{n \Var X_1} \sqrt{\log \frac1\delta}
\] for some constant $C$.

\section*{Roadmap}
We'll study Chernoff and Cram\'er bounds, which lead to Hoeffding's,
Bennett's and Bernstein's inequalities.

We'll look at the coupon collector's problem.

Another question with applications in bioinformatics and
\texttt{git diff}ing is the following.
\begin{question}
    Given two iid finite sequences $(X_1, \dots, X_n)$ and
    $(Y_1, \dots, Y_n)$, what is the length of a longest common subsequence?

    A \emph{common subsequence} of length $k$ is a sequence of length $k$
    that is a (not necessarily contiguous) subsequence of both $X$ and $Y$.
\end{question}

\begin{theorem}[Talagrand's principle] \label{thm:tal}
    A random variable that depends smoothly on a large number of random
    variables satisfies Chernoff-type bounds.
\end{theorem}

\subsection*{Methods}
\begin{itemize}
    \item Chernoff-type bounds
    \item Tensorization techniques to break a function of
    random variables into its constituents using the
    \begin{enumerate*}
        \item martingale,
        \item Effron-Stein, and
        \item entropy methods.
    \end{enumerate*}
    \item Applying isoperimetric inequalities to probability spaces.
    \item Transportation method
\end{itemize}

\subsection*{Applications}
\begin{itemize}
    \item Johnson-Lindenstrauss lemma
    \item Hypercontractivity
    \item Bounds on the performance of LASSO
    \item Blowing-up lemmas
    \item Bin-packing, coupon collector, birthday paradox, LCS, et cetera.
\end{itemize}
