\chapter*{The Course}
\lecture{2025-08-04}{}

\textbf{Textbook:} M.~A.~Armstrong's \textit{Basic Topology}. \\
\textbf{Teams Code:} \texttt{xr80svy}

\begin{center}
    When are two topological spaces \emph{equivalent}?
\end{center}
The strongest notion of equivalence is homeomorphism.
This preserves all properties pertaining to continuity.

However, visually, this is asking for too much.
$\R^2 \setminus \set 0$ and $S^1$ are not homeomorphic.
However, thinking loosely in terms of deformations, both these spaces
hace one ``hole''.
If we are allowed to deform space to any extent allowed \emph{without
creating or destroying holes}, one can switch between these two spaces.

The goal of algebraic topologically is to associate an algebraic object to
each topological space such that two topological spaces are equivalent,
in the second sense, iff their algebraic structures are isomorphic.

Algebraic topology is thus the study of algebraic invariants under homotopy
equivalence of spaces.

\chapter{The Fundamental Group} \label{chp:1}
Let $X$, $Y$ be topological spaces.
The word ``map'' without qualification will refer to continuous maps.
\begin{definition*}[Homotopy] \label{def:homotopy}
    Maps $f, g\colon X \to Y$ are \emph{homotopic} iff there exists a map
    $F\colon X \times I \to Y$ such that $F(\cdot, 0) = f$ and
    $F(\cdot, 1) = g$.
    $F$ is called a \emph{homotopy} from $f$ to $g$ and we write $f \sim g$.
\end{definition*}
For example, the maps $f(s) = e^{2\pi i s}$ and the cardioid
$g(s) = 2 (1 - \cos (2\pi s)) e^{2\pi i s}$ are homotopic.

\begin{remark}
    If $f \equiv g$ on some subset $A \subseteq X$ and
    $F(a, \cdot) \equiv f(a) = g(a)$ for all $a \in A$, we say that
    $f$ and $g$ are homotopic relative to $A$.
\end{remark}

\begin{examples}
    \item If $C$ is a convex subset of $\R^n$, then any two maps from any
        space $X$ are homotopic.
        A homotopy is given by $F(\cdot, t) = (1 - t) f + t g$.
        \todo{check that this is continuous}
        If $f \equiv g$ on $A \subseteq X$, then $F(a, t) = f(a) = g(a)$ on
        $A$.
    \item The other way does not work.
        The maps $f \equiv 0$ and $g \equiv 1$ from $[0, 1]$ to $\set{0, 1}$
        are not homotopic.
        Any homotopy would have connected domain but disconnected range.

        What if $Y$ were path-connected?
        Any two maps from a convex set to $Y$ would be homotopic.
        This follows from the next example.
    \item Let $C$ be convex and $Y$ be any space.
        Then any map $f\colon C \to Y$ is homotopic to some constant map.
        To construct a homotopy, pick any point $p \in C$ and define
        $F(x, t) = f((1 - t) x + t p)$.
    \item Let $f, g\colon X \to S^n \subseteq \R^{n+1}$ such that $f + g$
        is never $0$.
        Then $f \sim g$.
\end{examples}
