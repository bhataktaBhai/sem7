\lecture{2025-08-18}{}
\begin{examples}
    \item Square root mod: Given $a, b, c \in \Z_+$, check if there exists a
        natural number $x \le c$ such that $x^2 = a \pmod b$.
        This is $\NP$-complete.
    \item A variant of integer factoring:
        Given $L, U, N \in \Z_+$, check if there exists a 
        \emph{natural number} $d \in [L, U]$ such that $d \mid N$.
        This is $\NP$-hard under randomized Karp reductions.
    \item Minimum circuit size problem (MCSP):
        Given the truth table of a Boolean function $f$ and an integer $s$,
        check if there is a circuit of size at most $s$ that computes $f$.
        This is in $\NP$.

        Is it $\NP$-complete? Who knows!?
        Levin (or was it Cook?) delayed his publication by a year or two
        trying to prove that it is, but we don't have an answer yet.
        [ILO20] showed that the multi-output version is $\NP$-hard under
        polynomial-time randomized reductions.
        [Hira22] showed the same for the partial function version.
\end{examples}

\begin{theorem}[Independent set] \label{thm:is-npc}
    Deciding whether a graph $G$ has an independent set of size $k$ is
    $\NP$-complete.
\end{theorem}
We will reduce $\eSAT$ to $\INDSET$.
\begin{proof}
    Let $\phi$ be a $3$CNF with $m$ clauses and $n$ variables.
    Assume that every clause has exactly $3$ literals.
    Associate with each such clause a clique of size $2^3 - 1 = 7$.
    Each vertex denotes a partial assignment: an assignment of the three
    literals that satisfies this clause.

    This gives $m$ cliques $C_1, \dots, C_m$ consisting of $7$ vertices
    each.
    Each vertex denotes a partial assignment to the literals involved.
    Draw an edge between any two vertices which correspond to incompatible
    Boolean assignments.
    Call the resultant graph $G$.

    $\phi$ is satisfiable iff $G$ has an independent set of size $m$.
    \begin{itemize}
        \item If $G$ has such an independent set, it must have exactly one
            vertex from each of $C_1, \dots, C_k$.
            Since these are all compatible with each other, this yields
            a global assignment.
            Since each partial assignment satisfies the corresponding
            clause, so does the global assignment.
    \end{itemize}
\end{proof}

\begin{theorem}[Clique] \label{thm:clique-npc}
    Deciding whether a graph $G$ has a $k$-clique is $\NP$-complete.
\end{theorem}
\begin{proof}
    $G$ has a $k$-clique iff $\wbar{G}$ has an independent set of size
    $k$.
    Thus $\INDSET$ reduces to this problem.
\end{proof}

\begin{theorem}[Vertex cover] \label{thm:vertex-cover-npc}
    Deciding whether a graph $G$ has a vertex cover of size $k$ is
    $\NP$-complete.
\end{theorem}
\begin{proof}
    $G$ has a vertex set of size $k$ iff $G$ has an independent set of size
    $n - k$.
    Again $\INDSET$ reduces to this problem.
\end{proof}

\begin{theorem}[$0/1$ integer programming] \label{thm:bit-programming-npc}
    A $0/1$ integer program is a set of affine inequalities with rational
    coefficients where the variable lie in $\set{0, 1}$.

    The set of satisfiable $0/1$ integer programs is $\NP$-complete.
\end{theorem}
\begin{proof}
    $\eSAT$ reduces naturally to bit programming.
    A clause $x_1 \lor \neg x_2 \lor x_3$ maps to the inequality
    $x_1 + (1 - x_2) + x_3 \ge 1$.
\end{proof}

\begin{theorem}[Max cut] \label{thm:max-cut-nph}
    Given a graph, finding a cut with the maximum size is $\NP$-hard.
\end{theorem}
We are not stating that it is $\NP$-complete because, the way we have stated
it, $\NP$ only consists of decision problems.
\begin{proof}
    We showed that the decision problem $\mathsf{VCover}$ is $\NP$-hard.
    Thus the optimization version $\mathsf{MinVCover}$, finding a minimal
    vertex cover, is $\NP$-hard.
    We will now reduce $\mathsf{MinVCover}$ to $\mathsf{MaxCut}$.
\end{proof}
\todo{I left the lecture room at this point.}
