\lecture{2025-08-19}{}
A converse of \cref{thm:base} is also true.
\begin{theorem*} \label{thm:base:converse}
    Let $\mcB$ be a base for a topology $\T$ on $X$.
    Then a set $V \subseteq X$ is open iff for all $x \in V$, there exists
    a $W \in \mcB$ such that $x \in W \subseteq V$.
\end{theorem*}
\begin{proof}
    Let $V$ be open.
    Then, since $\mcB$ is a basis, we simply write $V$ as a union of open
    sets, and thus obtain a $W$ for every $x \in V$.

    Now suppose that for every $x \in V$ there exists a $W_x \in \mcB$ with
    $x \in W_x \subseteq V$.
    Then $V = \bigcup_{x \in V} W_x$ is open as each basic open set is open.
\end{proof}

\begin{question}
    Let $\mcB \subseteq 2^X$.
    When is $\mcB$ a base for \emph{some} topology on $X$?
    That is, when is \[
        \T = \set{\bigcup_{\alpha \in I} B_\alpha
            : \set{B_\alpha}_\alpha \subseteq \mcB}
    \] a topology?
\end{question}
It is easy to see that the resultant $\T$ is closed under arbitrary unions.
It also contains the null set.
We only require that it is closed under finite intersections,
and contains $X$.
Thus we need:
\begin{itemize}
    \item $\bigcup \mcB = X$.
    \item For $V, W \in \T$, $V \cap W$ ``must'' belong to $\T$.
        Write $V = \bigcup_{\alpha \in I} V_\alpha$ and
        $W = \bigcup_{\beta \in J} W_\beta$ where $V_\alpha$ and $W_\beta$
        are in $\mcB$.

        Now \begin{align*}
            V \cap W &= \bigcup_{(\alpha, \beta) \in I \times J} V_\alpha \cap W_\beta.
        \intertext{\Cref{thm:base:converse} shows that, since
        $V_\alpha \cap W_\beta$ must be open, we ``must'' have for every
        $x \in V_\alpha \cap W_\beta$ the existence of a $U_x$ such
        that $x \in U_x \subseteq V_\alpha \cap W_\beta$.
        In that case,}
            V \cap W &= \bigcup_{(\alpha, \beta) \in I \times J}
                \ab(\bigcup_{x \in V_\alpha \cap W_\beta} U_x)
        \end{align*} would be in $\T$.
\end{itemize}

This yields the following criterion.
\begin{theorem*}[Base criterion] \label{thm:base-criterion}
    $\mcB \subseteq 2^X$ is a base for a topology on $X$ iff
    \begin{enumerate}
        \item $\bigcup \mcB = X$, and
        \item for any $V, W \in \mcB$ and $x \in V \cap W$, there exists a
            $U \in \mcB$ such that $x \in U \subseteq V \cap W$.
    \end{enumerate}
\end{theorem*}
\begin{proof}
    Let $\T$ be the set of unions of elements in $\mcB$.
    If $\mcB$ is a base, both the conditions hold easily
    by \cref{thm:base:converse}.

    For the converse, assume that the conditions hold.
    $\O, X \in \T$ trivially.
    Closure under arbitrary unions holds trivially.
    Write $V = \bigcup_{\alpha \in I} V_\alpha$ and
    $W = \bigcup_{\beta \in J} W_\beta$ where $V_\alpha$ and $W_\beta$
    are in $\mcB$.
    Now fix $x \in V \cap W$.
    Then $x \in V_{\alpha^*} \cap W_{\beta^*}$ for some
    $(\alpha^*, \beta^*) \in I \times J$.
    Thus, by the second hypothesis, there exists a $U \in \mcB$ such that
    $x \in U \subseteq V_{\alpha^*} \cap W_{\beta^*} \subseteq V \cap W$.
    By \cref{thm:base}, $\mcB$ is a base.
\end{proof}

\begin{examples}
    \item On $\R^2$, let $\mcB = \set{(a, b) \times (c, d) : a, b, c, d \in \R}$.
        The union of these is $\R^2$.
        Let $(a_1, b_1) \times (c_1, d_1)$ and
        $(a_2, b_2) \times (c_2, d_2)$ be in $\mcB$.
        Their intersection is \[
            (\max\set{a_1, a_2}, \min\set{b_1, b_2})
                \times (\max\set{c_1, c_2}, \min\set{d_1, d_2}) \in \mcB.
        \] Thus by \cref{thm:base-criterion}, $\mcB$ is a base on $\R^2$.
    \item{} [Metric spaces]
        Let $(X, d)$ be a metric space
        and $\mcB = \set{B(x; r) : x \in X, r \in (0, \infty)}$.
        \begin{theorem}
            $\mcB$ is a base for a topology on $X$.
        \end{theorem}
        \begin{proof}
            $\bigcup \mcB = X$ as each $x \in B(x; 1)$.
            Let $V_1 = B(x_1; r_1)$ and $V_2 = B(x_2; r_2)$.
            For any $x \in V_1 \cap V_2$,
            let $r = \min\set{r_1 - d(x, x_1), r_2 - d(x, x_2)}> 0$.
            Then for any $y \in B(x; r)$, we have that \[
                d(y, x_1) \le d(y, x) + d(x, x_1)
                    < r_1 - d(x, x_1) + d(x, x_1) = r_1.
            \]
            Similarly $d(x_2, y) < r_2$.
            Thus $y \in V_1 \cap V_2$.
            Since $y$ was arbitrary,
            $B(x; r_1) \subseteq V_1 \cap V_2$.
        \end{proof}
        We did not use positivity (distinct points having a positive
        distance) in this proof.
        Thus, such pseudometrics still generate a nice topology.
\end{examples}

\paragraph{``Closeness'' in arithmetic/algebra vs analysis/geometry.}
\begin{itemize}
    \item For $r \in (0, \infty)$, $x, y \in \R$ are $r$-close if
    $\abs{x - y} < r$.
    That is, one cannot tell apart $x$ and $y$ at resolution $r$.
    But in an algebraic or purely arithmetic context,
    we do not have inequalities.
    \item For $n, m \in \Z$ and $d \in \Z^+$, we say $n, m \in \Z$ are
    $d$-close if $d \mid (n - m)$, that is, $[n] = [m] \in \Z/d\Z$.
    $d$-closeness is transitive in this case.
    This motivates the arithmetic progression topology.
\end{itemize}
\begin{proposition}[Arithmetic progression topology] \label{thm:ap-topo}
    Let $\mcB = \set{n + d \Z : n \in \Z, d \in \Z^+}$.
    This is a base on $\Z$.
\end{proposition}
\begin{proof}
    $\Z = 0 + 1 \Z \in \mcB$.

    For $n_1 + d_1 \Z$ and $n_2 + d_2 \Z$ in $\mcB$ and $x$ in their
    intersection, note that $x \in x + d_1 d_2 \Z$.
\end{proof}
Note that any set $n + d \Z$ is also closed.
That is, each basic open set is also a closed set.
But \[
    \Z \setminus \set{\pm 1} = \bigcup_{p \text{ prime}} p \Z
\] is open.
It cannot be closed, since $\set{\pm 1}$ cannot be open!
(Any non-empty open set is the union of infinite sets.)
If there were finitely many primes, this would be a finite intersection of
closed sets.
Hence, there are infinitely many primes (this is Furstenberg's proof of
the infinitude of primes).
