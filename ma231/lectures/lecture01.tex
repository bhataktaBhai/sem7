\chapter*{The course}
\lecture{2025-08-05}{}

\textbf{\href{https://math.iisc.ac.in/~gadgil/topology-2025/index.html}{Course website}}

\subsubsection{Course Details}
\begin{itemize}
    \item \textbf{Instructor:} \href{https://math.iisc.ac.in/~gadgil}{Siddhartha
      Gadgil}
    \item \textbf{E-mail:} \href{mailto:siddhartha.gadgil@gmail.com}{siddhartha.gadgil@gmail.com}
    \item \textbf{Office:} N-15, Department of Mathematics, IISc.
    \item \textbf{Timing:} Tue, Thu 8:30 -- 10:00 am.
    \item \textbf{Venue:} LH-1.
    \item \textbf{Teaching Assistants:} Sudip Dolai, Pallab Kumar Hembram
    \item \textbf{Office:} L-22, Department of Mathematics, IISc.
    \item \textbf{Tutorial timing:} TBA
    \item \textbf{Microsoft Teams Code:} \texttt{tbjeh8w}
\end{itemize}

\paragraph{Examinations and Grades.}\label{examinations-and-grades}
Grades will be based on the assignments, midterm examination and final
examinations with the following weightages.
\begin{itemize}
    \item[(10\%)] Assignments, posted roughly once a week.
    \item[(40\%)] Midterm
    \item[(50\%)] Final exam
\end{itemize}

\paragraph{Suggested books:}
\begin{enumerate}
    \item Armstrong,~M.~A., Basic Topology, Springer (India), 2004.
    \item Munkres,~J.~R., Topology, Pearson Education, 2005.
    \item Viro,~O.~Ya., Ivanov,~O.~A., Netsvetaev,~N., and Kharlamov,~V.~M.,
      Elementary Topology: Problem Textbook, AMS, 2008.
\end{enumerate}

We will mostly follow the book \textbf{Elementary Topology: Problem
Textbook} by \textbf{Viro et al.}, which is available online.

\paragraph{Additional Resources.}
This course was
\href{https://math.iisc.ac.in/~gadgil/topology-2021/index.html}{taught online}
in 2021.
The \href{https://math.iisc.ac.in/~gadgil/topology-2021/all-lectures/}{lectures}
are online as are the
\href{https://math.iisc.ac.in/~gadgil/topology-2021/Whiteboard.pdf}{whiteboards}.
If you want to printout the whiteboards, please use the
\href{https://math.iisc.ac.in/~gadgil/topology-2025/notes.pdf}{compact version}.

Note that students are responsible for all the material covered in the
lectures this semester, which is likely to be more than that in the
above resources.
Experience also suggests that offline lectures are more
effective.
Thus, it is wise to use the above as supplements, not
substitutes, for the lectures.

\paragraph{Syllabus.}
Open and closed sets, Continuous functions, Metric topology, Product
topology, Connectedness and path-connectedness, Compactness,
Countability axioms, Separation axioms, Complete metric spaces, Quotient
topology, Topological groups, Orbit spaces, Urysohn's lemma,
Metrizability, Baire Category theorem.

\begin{center}
    Study of properties associated with continuity and limits.
\end{center}
\begin{examples}
    \item Any continuous function $f\colon [0, 1] \to \set{0, 1}$ is
    constant, by the intermediate value theorem.
    \item Any continuous function $f\colon [0, 1] \to \R$ is bounded.
\end{examples}
The first example illustrates connectivity, and the second compactness.
We study continuous functions $f\colon X \to Y$ between topological spaces.
Topological spaces must come with a notion of continuity and limits.
They must coincide with our usual understanding in familiar settings such as
$\R^n$, convergence of sequences, and so on.
We should be able to construct new topological spaces from old ones.

\chapter{Topological spaces} \label{chp:topo}
Our first attempt is \emph{metric spaces}.
\begin{definition}[Metric space] \label{def:metric-space}
    A \emph{metric} on a set $X$ is a function
    $d\colon X \times X \to [0, \infty)$
    such that
    \begin{enumerate}[label=(M\arabic*)]
        \item $d(x, x) = 0$ for all $x \in X$,
        \item{} [Symmetry] $d(x_1, x_2) = d(x_2, x_1)$ for all
            $x_1, x_2 \in X$, and
        \item{} [Triangle inequality]
            $d(a, c) \le d(a, b) + d(b, c)$ for all $a, b, c \in X$.
        \item{} [Positivity] $d(x_1, x_2) = 0$ implies that
            $x_1 = x_2$.
    \end{enumerate}
    The pair $(X, d)$ is called a \emph{metric space}.
\end{definition}
\begin{remark}
    If positivity does not hold, then we can define $a \sim b$ on $X$ iff
    $d(a, b) = 0$.
    $d$ then induces a metric on the quotient space $X/{\sim}$.
\end{remark}
In the early 1900's, it wasn't clear whether the notion of a topological
space is even required.
In fact, Hausdorff introduced topological spaces in the second edition of
his textbook, but went back to metric spaces in the third edition.

Here is a non-example of a metric space.
Consider $X = \R^\R$ and say that $f_n \to f$ iff $f_n(x) \to f(x)$ for all
$x \in \R$.
\begin{claim} \label{thm:non-metric}
    There is no metric on $\R^\R$ such that $f_n \to f$ iff
    $d(f_n, f) \to 0$.
\end{claim}
\todo{We will prove this much later, once we study product topologies and
first countability.}

Suppose $(X, d)$ is a metric space and $\sim$ an equivalence relation on
$X$.
We want that any continuous function $f\colon X \to Y$ for which
$a \sim b$ implies $f(a) = f(b)$, the induced function
$\wtld{f}\colon X/{\sim} \to Y$ is continuous.
This is similar to how group homomorphisms induce a homomorphism on quotient
groups.
In general, such quotient metrics do not exist.
\todo{We will come back to this with Hausdorff things.}
\begin{examples}
    \item Consider $\R$ with $a \sim b$ iff $a - b \in \Q$.
    \item $\R^2$ with equivalence classes as Gadgil drew.
\end{examples}

The standard approach is to abstract properties of \emph{open sets}.
In the metric space $(X, d)$, a set $U \subseteq X$ is open if for every
$x \in U$ there exists an $\eps > 0$ such that the $\eps$-neighborhood
of $x$ is contained in $U$.
\begin{definition*}[Topological space] \label{def:topo}
    Let $X$ be a set.
    A \emph{topology} on $X$ is a collection $\T$ of subsets of $X$ such
    that
    \begin{enumerate}[label=(T\arabic*)]
        \item $\O, X \in \T$;
        \item if $\set{U_\alpha}_{\alpha \in I}$ is a subcollection of
            $\T$, then $\bigcup_{\alpha \in I} U_\alpha \in \T$;
        \item if $U_1, \dots, U_n \in \T$, then $U_1 \cap \dots \cap U_n \in \T$.
    \end{enumerate}
    Sets in $\T$ are called \emph{open sets} and the pair $(X, \T)$ is
    called a \emph{topological space}.
    Complements of open sets are called \emph{closed sets}.
\end{definition*}
\begin{examples}
    \item{} [Discrete topology] For any set $X$, $2^X$ is a topology on $X$.
    \item{} [Indiscrete topology] For any set $X$,
        $\set{\O, X}$ is a topology on $X$.
    \item{} [Standard topology on $\R$] The collection of all
        subsets that can be expressed as a union of open intervals
        is a topology on $\R$.
        Alternatively and more simply, the metric space definition also
        yields the same topology.
\end{examples}
