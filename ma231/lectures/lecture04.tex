\lecture{2025-08-14}{}
\begin{lemma} \label{thm:cantor-descent}
    For each $\alpha_1, \dots, \alpha_k, \alpha_{k+1} \in \set{0, 1}$,
    \begin{enumerate}
        \item $C_{\alpha_1, \dots, \alpha_k, \alpha_{k+1}} \in
            C_{\alpha_1, \dots, \alpha_k}$, and
        \item $C_{\alpha_1, \dots, \alpha_k, 0} \cap
            C_{\alpha_1, \dots, \alpha_k, 1} = \O$.
    \end{enumerate}
\end{lemma}
\begin{proof}
    Trivial.
\end{proof}

Let $C^{(k)} = \bigcup_{\alpha \in \set{0, 1}^{[k]}} C_{\alpha_1, \dots, \alpha_k}$.
This is a finite union of closed intervals, hence closed.
However, we also saw that $C \subseteq C^{(k)}$ for each $k$ via
\cref{eq:cantor-expansion}.
Thus $C \subseteq \bigcap_{k=1}^\infty C^{(k)}$.
\begin{proposition}
    $C = \bigcap_{k=1}^\infty C^{(k)}$.
\end{proposition}
\begin{proof}
    Let $x \in \bigcap_{k=1}^\infty C^{(k)}$.
    That is, for each $k$, there exists $\alpha \in 2^{[k]}$ such that
    $x \in C_{\alpha_1, \dots, \alpha_k}$.
    \begin{claim}
        For $l \ge k$, if $x \in C_{\alpha_1, \dots, \alpha_k}$ and
        $x \in C_{\beta_1, \dots, \beta_l}$, then
        $\alpha_1 = \beta_1, \dots, \alpha_k = \beta_k$.
    \end{claim}
    \begin{subproof}
        \Cref{thm:cantor-descent} implies that
        $C_{\beta_1, \dots, \beta_l} \subseteq C_{\beta_1, \dots, \beta_k}$.
        Its second part implies $\alpha_1 = \beta_1, \dots, \alpha_k = \beta_k$
        by looking at the first point of difference, should it exist.
    \end{subproof}
    Thus, there exists a unique sequence $\alpha_1, \alpha_2, \dots$ such
    that $x \in C_{\alpha_1, \dots, \alpha_k}$ for all $k$.
    \begin{claim}
        $x = \sum_{i=1}^\infty \frac{2\alpha_i}{3^i}$.
    \end{claim}
    \begin{subproof}
        For $k \ge 1$, $\sum_{i=1}^k \frac{2\alpha_i}{3^i} \in
        C_{\alpha_1, \dots, \alpha_k}$.
        But so is $x$.
        Thus \[
            \abs*{x - \sum_{i=1}^\infty \frac{2\alpha_i}{3^i}}
                \le \frac1{3^k}.
        \] Since this holds for all $k$, these two nubers are the same.
    \end{subproof}
    Since $x$ was arbitrary, $\bigcap_{k=1}^\infty C^{(k)} \subseteq C$.
\end{proof}
\begin{corollary}
    As an intersection of closed sets, $C$ is closed.
\end{corollary}

\section{Bases} \label{sec:bases}
We say that the natural topology $\T$ on $\R$ is the collection of unions
of open intervals.
We say that open intervals are a \emph{base} for $\T$.

\begin{definition*}[Base] \label{def:base}
    A collection of subsets $\mcB$ of $X$ is said to be a base for a
    topology $\T$ on $X$ if \[
        \T = \set{\bigcup_{\alpha \in I} B_\alpha :
                \set{B_\alpha} \subseteq \mcB}.
    \] Elements of $\mcB$ are called \emph{basic open sets}.
\end{definition*}

\begin{question} \leavevmode
    \begin{itemize}
        \item Is there a more convenient characterization of the bases for
            a topology?
        \item When is a collection $\mcB \subseteq 2^X$ a base for
            \emph{some} topology?
        \item When are two collections $\mcB_1$ and $\mcB_2$ bases for the
            same topology?
    \end{itemize}
\end{question}

\begin{examples}
    \item A base for the discrete topology on $X$ is $\mcB = \binom{X}{1}$.
        This is the minimum (not just minimal) base for the
        discrete topology.
    \item A base for the indiscrete topology on $X$ is $\mcB = \set X$.
        This is one of the only two bases, the other being $\set{\O, X}$.
    \item A base for the cofinite topology on any infinite set $X$ is \[
        \mcB = \set{X \setminus F : 2^{2^{100}} < \abs{F} < \infty}.
    \]
    \item A base for the topology on $\R$ is \[
            \set{(a, b) : a, b \in \Q}.
        \] This is because any interval with real (or infinite) endpoints can
        be written as the union of intervals with rational endpoints tending
        towards those.
        This depends on \cref{thm:base}\labelcref{thm:base:basic}
        that follows.
\end{examples}

\begin{theorem*} \label{thm:base}
    Let $\T$ be a topology on $X$ and $\what{\mcB} \subseteq 2^X$.
    \begin{enumerate}
        \item $\what{\mcB}$ is a base for the topology $\mcT$ if every
            $V \in \what{\mcB}$ is open and for every $V \in \T$ and
            $x \in V$, there exists a $W \in \what{\mcB}$ such that
            $x \in W \subseteq V$. \label{thm:base:open}
        \item Let $\mcB$ be a given base for $\T$.
            Then $\what{\mcB}$ is a base for $\T$ if every
            $V \in \what{\mcB}$ is open and for all $V \in \mcB$ and
            $x \in V$, there exists a $W \in \what{\mcB}$ such that
            $x \in W \subseteq V$. \label{thm:base:basic}
    \end{enumerate}
\end{theorem*}
\begin{proof}
    % We will only prove the ``if'' part of both statements.
    Assume that every $V \in \what{\mcB}$ is open.
    Since each $V \in \what{\mcB}$ is open, their unions are in $\T$.
    \begin{enumerate}
        \item Assume that for all $x \in V \in \T$, there is a neighborhood
            $W \in \what{\mcB}$ of $x$ that is contained within $V$.

            For any $V \in \T$, simply write $V = \bigcup_{x \in V} W_x$
            where $W_x$ is gotten from the hypothesis.
        \item Assume the requisite hypothesis once more.
            For any $V \in \T$, write it as
            $\bigcup_{\alpha \in I} B_\alpha$ for a
            $\set{B_\alpha} \subseteq \mcB$.
            Further write each $B_\alpha$ as
            $B_\alpha = \bigcup_{x \in B_\alpha} W_x$. \qedhere
    \end{enumerate}
\end{proof}
