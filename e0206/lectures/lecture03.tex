\lecture{2025-08-13}{Independent sets; Ramsey numbers; Colorings}
\begin{theorem} \label{thm:is-lb}
    A graph of average degree $d$ has an independent set of size at
    least $n/2d$.
\end{theorem}
\begin{proof}
    Fix $p \in [0, 1]$ to be optimized over later.
    Pick each vertex independently with probability $p$.
    Call this set $S$.
    For each edge in edge, (simultaneously) delete an arbitrary vertex in
    $S$.
    Call the resultant set $S'$.

    The expected size of $S$ is $np$.
    The expected number of edges is $\abs{E} p^2$.
    Thus the expected value of $\abs{S} - \abs{S'}$ is at most
    $\abs{E} p^2$. \[
        \EA{S'} \ge np - \frac{nd}{2} p^2.
    \] The lower bound is maximised by $p = \frac1d$, yielding
    $\EA{S'} \ge \frac{n}{2d}$.
\end{proof}

\begin{exercise}
    If $G$ has maximum degree $d$, come up with a simple greedy algorithm
    to output an independent set of size $n/(d+1)$.
\end{exercise}
\begin{solution}
    Start with $S = \O$.
    Pick any vertex and add it to $S$.
    Delete all its neighbours.
    Continue to exhaustion.
\end{solution}

\begin{theorem}
    $R(k, k) \ge \frac1e k 2^{k/2}$.
\end{theorem}
\begin{proof}
    Color the edges of $K_n$ uniformly at random.
    From each monochromatic clique of size $k$, delete one arbitrary vertex.

    For each clique of size $k$, the probability that it is monochromatic is
    $2^{1 - \binom{k}{2}}$.
    The resultant set of vertices is expected to have size at least \[
        n - \binom{n}{k} 2^{1 - \binom{k}{2}} \ge \wtld{n} \coloneq
            n - \frac{n^k}{k!} 2^{1 - \binom{k}{2}}.
    \] Thus we have the existence of a ``nice'' coloring of $K_{\wtld{n}}$,
    so that $R(k, k) > \wtld{n}$.

    To optimise over $n$, we differentiate and select a large $n$ so that \[
        1 - k \frac{n^{k-1}}{k!} 2^{1 - \binom{k}{2}} > 0.
    \] Using Stirling's approximation as in the proof of \cref{thm:ramsey1},
    we get that
    \[
        n < \frac1{2^{\frac1{k-1}}} 2^{k/2} \frac {k-1}e
    \] suffices.
    The first term again approaches $1$ from above, so we ignore it.

    Substituting this in $\wtld{n}$ gives \begin{align*}
        \wtld{n} &= \frac ke 2^{k/2} - \ab(\frac ke)^k \frac{2^{k^2}}{k!} 2^{1 - \binom{k}{2}} \\
            &\ge \frac ke 2^{k/2} - 2^{1 + \frac{k^2 + k}{2}} \frac1e
    \end{align*}
    \todo{what?}
\end{proof}

What does a graph $G$ with a large chromatic number $\chi(G)$?
One imagines a densely connected subgraph.
But such a subgraph would have short cycles.
\begin{theorem}
    For any $k$ and $l$, there exists a graph $G$ with chromatic number
    more than $k$ and girth more than $l$.
\end{theorem}
To bound the chromatic number below, we can bound the size of the largest
independent set above.
\Cref{thm:is-lb} gives a lower bound and is not useful here.
% \begin{proof}
%     Let $G = G(n, p)$ be an Erd\"os-R\'enyi random graph with $n$ and $p$ to
%     be chosen later.
%     For any $i$, the expected number of independent sets of size $i$ is
%     given by $\binom{n}{i} (1 - p)^{\binom{i}{2}}$.
%     For sufficiently small $p$, we can make it so that this quantity is less
%     than $1$ when $i = n/k$, meaning that for some particular instance of
%     $G$, the chromatic number is more than $k$.
%
%     The girth is trickier to handle.
%     The expected number of cycles of length $j$ is
%     $\binom{n}{j} \frac{(j-1)!}{2} p^j$.
%     Thus the expected number of cycles of length at most $l$ is
%     \begin{align}
%         \sum_{j=3}^l \binom{n}{j} \frac{(j-1)!}{2} p^j
%             &\le \sum_{j=3}^l \frac{n^j}{j!} \frac{(j-1)!}{2} p^j \\
%             &\le \frac l6 \frac1{1-np}.
%     \end{align}
%     If we remove an arbitrary vertex from each cycle of length at most $l$,
%     we get a graph of girth more than $l$.
%     However, this may decrease the chromatic number of the graph.
%
%     The key observation is that it will not decrease the size of the largest
%     independent set.
%     (This is why we are adamant on deleting vertices as opposed to edges.)
%     Once this deletion process is complete, we will end up with \[
%         n' \coloneq n - \frac l6 \frac1{1-np}
%     \] vertices and we require that for $i = n'/k$, \[
%         \binom{n'}{i} (1 - p)^{\binom{i}{2}} < 1.
%     \]
%     For $\E[n']$ to be at least, say, $n/2$, we require \[
%         p \le \frac1n - \frac1{3nl}
%     \]
%     We bound \begin{align*}
%         \binom{n'}{i} (1 - p)^{\binom{i}{2}}
%             &\le (n')^i (1 - p)^{\binom{i}{2}} \\
%             &\le \ab(n' (1 - p)^{(i - 1) / 2})^i \\
%             &\le \ab(n'
%     \end{align*}
%     \todo[inline]{Optimize}
% \end{proof}
