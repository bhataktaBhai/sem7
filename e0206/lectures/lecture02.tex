\lecture{2025-08-11}{Vector sums; Ramsey numbers; intersecting families}
\begin{theorem}
    Given unit vectors $v_1, \dots, v_n \in \R^n$, there exist
    $\eps_1, \dots, \eps_n \in \set{\pm 1}$ such that \[
        \norm*{\sum_i \eps_i v_i} \le \sqrt n.
    \]
\end{theorem}
\begin{proof}
    Set the $\eps_i$s to be iid with
    $\P\set{\eps_1 = 1} = \P\set{\eps_1 = -1} = \frac12$.
    Let \begin{align*}
        X \coloneq \norm*{\sum_i \eps_i v_i}^2
            &= \sum_i \norm{v_i}^2 + \sum_{i,j} \eps_i \eps_j \innerp{v_i}{v_j} \\
            &= n + \sum_{i,j} \eps_i \eps_j \innerp{v_i}{v_j}.
    \end{align*}
    Then $X$ has expectation $n$, so there exists a confirguration in which
    it is at most $n$.
\end{proof}
The other inequality also holds by the same proof.
The norm can be made at least $\sqrt n$.

\begin{exercise}
    Give a deterministic algorithm to yield $\eps_1, \dots, \eps_n$.
\end{exercise}
\begin{solution}
    Choose $\eps_1, \eps_2, \dots$ one by one, at each step choosing the
    option which yields the smaller current norm.
    The proof that this works is the same as the derandomization for
    half-cut shown in \cref{sec:derandomization}.
\end{solution}
\todo[inline]{Prof.~Anand says there is a simpler algorithm.}

\begin{definition}[Ramsey numbers] \label{def:ramsey-numbers}
    $R(k, l)$ is defined to be the smallest $n$ such that any coloring of
    the edges of $K_n$ using the colors Red and Blue has either a Red clique
    of $k$ vertices or a Blue clique of $l$ vertices.
\end{definition}
\begin{exercise}
    Show that $R(k, l)$ is finite.
    Ideally, show that $R(k, k) \le 4^k$.
\end{exercise}
% \begin{proof}
%     It suffices to bound the diagonal Ramsey numbers, since
%     $R(k, l) \le R(k \lor l, k \lor l)$.
%
%     For simplicity of notation let $n = R(k, k)$.
%     Note that $K_{2n}$ must have at least $2$ monochromatic cliques
%     If these have 
%     Consider $K_{R(k, k) + t}$ where $t$ is to be determined later.
%     Assume that this has no monochromatic clique of size $k + 1$.
%     WLOG let the first $k$ vertices form a Red clique.
%     Then each of the last $t$ vertices must have a Blue edge to one of
%     the first $k$.
%     If $t \ge k^{R(k, k)}$, then there is at least one $i \in [k]$ that has
% \end{proof}

\begin{examples}
    \item $R(1, 1) = 1$.
    \item $R(2, 2) = 2$.
    \item $R(3, 3) = 6$.
    \begin{proof}
        Let $v \in K_n$ be arbitrary.
        At least $3$ of its outgoing edges are the same color.
        Assume that to be Red.
        Consider the clique formed by the three corresponding neighbors.
        If all of the edges are Blue, we are done.
        If any of them is Red, it forms a Red clique together with $v$.

        It is easy to color $K_5$ such that it avoids $3$-cliques.
    \end{proof}
\end{examples}

\begin{theorem} \label{thm:ramsey1}
    For any $k$, $R(k, k) > \frac1{e\sqrt 2} k 2^{k/2}$.
\end{theorem}
\begin{proof}
    Let $n \in \N$.
    To show that $R(k, k) > n$, we must obtain a ``nice'' coloring of
    $K_n$, that is, one that avoids monochromatic $k$-cliques.

    Color the edges of $K_n$ uniformly at random.
    Then for any $k$-clique, the probability that it is monochromatic is
    $2 \ab(\frac12)^{\binom{k}{2}}$.

    Summing this up over all possible $k$-cliques yields an upper bound of
    $2 \binom{n}{k} \ab(\frac12)^{\binom{k}{2}}$ on the probability that
    there exists any monochromatic $k$-clique at all.
    If this bound is strictly less that $1$, then with some positive
    probability, we obtain a nice coloring of $K_n$.
    This happens when \[
        \binom{n}{k} < 2^{\binom{k}{2} - 1}.
    \] Now $\binom{n}{k} \le \frac{n^k}{k!} < 2^{\binom{k}{2} - 1}$ when
    $n < \frac1{2^{1 / k}} 2^{\frac{k - 1}{2}} k!^{\frac1k}$.

    Using that $k! \ge \sqrt{2 \pi k} \ab(\frac ke)^k$, we have that \[
        n < \ab(\frac{\pi k}{2})^{\frac1{2k}} \frac1{\sqrt 2} 2^{\frac k2} \frac ke
    \] suffices.
    The first term approaches $1$ from above, so we will write the bound as
    \[
        n < \frac1{e \sqrt 2} k 2^{\frac k2}.
    \]
    Thus, $R(k, k) \ge (e \sqrt 2)^{-1} k 2^{k/2}$.
\end{proof}

\begin{definition}[$k$-intersecting family] \label{def:intersecting-family}
    A family $F = \set{S_1, \dots, S_m}$ of subsets from a universe $U$ of
    $n$ elements is a \emph{$k$-intersecting family} if
    \begin{itemize}
        \item $\abs{S_i} = k$ for every $i \in [m]$, and
        \item $S_i \cap S_j \ne \O$ for every $i, j \in [m]$.
    \end{itemize}
\end{definition}

\begin{theorem} \label{thm:intersecting-family}
    Let $\abs{U} = n$ and $F$ be a $k$-intersecting family from $U$
    where $n \ge 2k$.
    Then $\abs{F} \le \binom{n-1}{k-1}$.
\end{theorem}
This bound is tight.
One such $F$ would be $\set{A \cup \set u : A \in \binom{U \setminus \set u}{k-1}}$,
where $u$ is an arbitrary element of $U$.
\begin{lemma}
    Let $\abs{U} = n$ and $F$ be a $k$-intersecting family from $U$
    where $n \ge 2k$.
    Fix a permutation $\sigma$ of $[n]$.
    Let $A_{\sigma, i} = \set{\sigma(i), \dots, \sigma(i+k-1)}$.
    (The addition is modulo $n$.)
    Then there are at most $k$ values of $i$ for which $A_{\sigma,i} \in F$.
\end{lemma}
\begin{proof}
    The permutation is irrelevant here.
    WLOG assume that it is the identity.
    Each $A_i$ has size $k$.

    Assume $A_i \in F$.
    The sets which overlap with $A_i$ are $A_{i-k+1}$ through
    $A_{i+k-1}$.
    That is, $2k - 2$ other sets which could be a part of $F$.
    However, $A_{i-k+1}$ and $A_{i+1}$ do not overlap.
    Neither do $A_{i-k+2}$ and $A_{i+2}$, all
    the way up to $A_{i-1}$ and $A_{i+k-1}$.
    Since only one set may be present from each of these pairs,
    only $k - 1$ of these ``other sets'' may be a part of $F$.
\end{proof}

\begin{proof}[Proof of \cref{thm:intersecting-family}]
    Sample $\sigma \in S_n$ and $i \in [n]$ uniformly at random.
    Note that $A_{\sigma, i}$ is then a uniform random subset of size $k$.
    Call this $A$.

    Thus $\P\set{A \in F} = \abs{F} \binom{n}{k}^{-1}$.
    However, this probability is also \begin{align*}
        \P(A \in F)
            &= \sum_\sigma \frac1{n!} \P(A \in F \mid \sigma) \\
            &\le \sum_\sigma \frac1{n!} \frac kn
    \end{align*}
    by the lemma.
    Thus $\abs{F} \le \binom nk \frac kn = \binom{n-1}{k-1}$.
\end{proof}
